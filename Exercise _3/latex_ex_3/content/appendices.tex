\begin{appendix}
\addappheadtotoc
\section{CPP CUDA Code - Basic Cuda - Task 1a}
\label{app_1a}

\begin{lstlisting}[language=C++, title=C++ Listing]
#include <stdio.h>
#include "timer.hpp"
int main(void)
{
	int k=0, i=8;
	int N_values[i] = {1000000, 2000000, 4000000, 8000000, 16000000, 32000000, 64000000, 128000000};
	double *gpu_x;
	float t_malloc=0, t_free=0;
	Timer timer;
	printf("\nsize, malloc, free\n");
	while(k < i) {
		int N = N_values[k];
		for (int n=0; n<5; n++) {
			timer.reset();
			cudaMalloc(&gpu_x, N*sizeof(double)); 
			cudaDeviceSynchronize();
			t_malloc += timer.get();
			timer.reset();
			cudaFree(gpu_x);
			cudaDeviceSynchronize();
			t_free += timer.get();
		} 
		printf("%d,%g,%g\n", N, 0.2*t_malloc, 0.2*t_free);
		k++;
	}
	return EXIT_SUCCESS;
}
\end{lstlisting}
\pagebreak

\section{CPP CUDA Code - Basic Cuda - Task 1b}
\label{app_1b}
\begin{lstlisting}[language=C++, title=C++ Listing]
#include <stdio.h>
#include "timer.hpp"

__global__ void initVec(double *vec1, double *vec2, int N) {
	unsigned int total_threads = blockDim.x * gridDim.x;
	unsigned int global_tid = blockIdx.x * blockDim.x + threadIdx.x;
	
	for (unsigned int i = global_tid; i<N; i += total_threads) {
		vec1[i] = (double)(i);
		vec2[i] = (double)(N-i-1);
	}
}

int main(void)
{
	int k=0, i=6;
	int N_values[i] = { 1000000, 2000000, 4000000, 8000000, 16000000, 32000000 };
	double *x, *y, *gpu_x, *gpu_y;
	Timer timer;	
	float option_1=0;
	k = 0;
	printf("\nsize,option1\n");
	while(k < i) {
		int N = N_values[k];
		for (int n=0; n<5; n++) {
			timer.reset();
			x = (double*)malloc(N*sizeof(double));
			y = (double*)malloc(N*sizeof(double));
			cudaMalloc(&gpu_x, N*sizeof(double)); 
			cudaMalloc(&gpu_y, N*sizeof(double));
			initVec<<<256, 256>>>(gpu_x, gpu_y, N);
			cudaDeviceSynchronize();
			option_1 += timer.get();
		}
		printf("%d,%g\n", N, 0.2*option_1);
		cudaFree(gpu_x);
		cudaFree(gpu_y);
		free(x);
		free(y);
		k++;		
	}
	float option_2=0;
	k = 0;
	printf("\nsize,option2\n");
	while(k < i) {
		int N = N_values[k];
		for (int n=0; n<5; n++) {
			timer.reset();
			cudaDeviceSynchronize();
			x = (double*)malloc(N*sizeof(double));
			y = (double*)malloc(N*sizeof(double));
			for (int i = 0; i < N; i++) {
				x[i] = (double)(i);
				y[i] = (double)(N-i-1);
			}
			cudaMalloc(&gpu_x, N*sizeof(double)); 
			cudaMalloc(&gpu_y, N*sizeof(double));
			cudaMemcpy(gpu_x, x, N*sizeof(double), cudaMemcpyHostToDevice);
			cudaMemcpy(gpu_y, y, N*sizeof(double), cudaMemcpyHostToDevice);
			cudaDeviceSynchronize();
			option_2 += timer.get();
		}
		printf("%d,%g\n", N, 0.2*option_2);
		cudaMemcpy(x, gpu_x, N*sizeof(double), cudaMemcpyDeviceToHost);
		cudaMemcpy(y, gpu_y, N*sizeof(double), cudaMemcpyDeviceToHost);
		cudaFree(gpu_x);
		cudaFree(gpu_y);
		free(x);
		free(y);
		k++;
	}
	return EXIT_SUCCESS;
}
\end{lstlisting}
\pagebreak

\section{CPP CUDA Code - Basic Cuda - Task 1c, 1d, 1e}
\label{app_1cde}
\begin{lstlisting}[language=C++, title=C++ Listing]
#include <stdio.h>
#include "timer.hpp"
__global__ void addVec(double *x, double *y, double *z, int N) {
	unsigned int total_threads = blockDim.x * gridDim.x;
	unsigned int global_tid = blockIdx.x * blockDim.x + threadIdx.x;
	for (unsigned int i = global_tid; i<N; i += total_threads) {
		z[i] = x[i] + y[i];
	}
}
int main(void)
{
	// Task c //
	double *x, *y, *z, *gpu_x, *gpu_y, *gpu_z;
	Timer timer;
	int N = 100;
	x = (double*)malloc(N*sizeof(double));
	y = (double*)malloc(N*sizeof(double));
	z = (double*)malloc(N*sizeof(double));
	for (int i = 0; i < N; i++) {
		x[i] = (double)(i);
		y[i] = (double)(N-i-1);
	}
	cudaMalloc(&gpu_x, N*sizeof(double)); 
	cudaMalloc(&gpu_y, N*sizeof(double));
	cudaMalloc(&gpu_z, N*sizeof(double));
	cudaMemcpy(gpu_x, x, N*sizeof(double), cudaMemcpyHostToDevice);
	cudaMemcpy(gpu_y, y, N*sizeof(double), cudaMemcpyHostToDevice);
	addVec<<<256, 256>>>(gpu_x, gpu_y, gpu_z, N);
	cudaMemcpy(z, gpu_z, N*sizeof(double), cudaMemcpyDeviceToHost);
	cudaFree(gpu_x);
	cudaFree(gpu_y);
	cudaFree(gpu_z);
	free(x);
	free(y);
	free(z);

	// Task d //
	int k = 0;
	int N_values[10] = { 100, 300, 1000, 3000, 10000, 30000, 100000, 300000, 1000000, 3000000 };
	printf("\nsize,time\n");
	while(k < 10) {
		float t_kernel=0;
		int N = N_values[k];
		x = (double*)malloc(N*sizeof(double));
		y = (double*)malloc(N*sizeof(double));
		z = (double*)malloc(N*sizeof(double));
		for (int i = 0; i < N; i++) {
			x[i] = (double)(i);
			y[i] = (double)(N-i-1);
		}
		cudaMalloc(&gpu_x, N*sizeof(double)); 
		cudaMalloc(&gpu_y, N*sizeof(double));
		cudaMalloc(&gpu_z, N*sizeof(double));
		cudaMemcpy(gpu_x, x, N*sizeof(double), cudaMemcpyHostToDevice);
		cudaMemcpy(gpu_y, y, N*sizeof(double), cudaMemcpyHostToDevice);
		timer.reset();
		for (int n=0; n<5; n++) {
			addVec<<<256, 256>>>(gpu_x, gpu_y, gpu_z, N);
			cudaDeviceSynchronize();
		}
		t_kernel += timer.get();
		printf("%d,%g\n", N, 0.2*t_kernel);
		cudaMemcpy(z, gpu_z, N*sizeof(double), cudaMemcpyDeviceToHost);
		cudaFree(gpu_x);
		cudaFree(gpu_y);
		cudaFree(gpu_z);
		free(x);
		free(y);
		free(z);
		k++;
	}
	// Task e //
	N = 10000000;
	k = 0;
	int params[7] = { 16, 32, 64, 128, 256, 512, 1024};
	printf("\nsqrt(threads),time\n");
	while(k < 7) {
		float t_kernel=0;
		int param = params[k];
		x = (double*)malloc(N*sizeof(double));
		y = (double*)malloc(N*sizeof(double));
		z = (double*)malloc(N*sizeof(double));
		for (int i = 0; i < N; i++) {
			x[i] = (double)(i);
			y[i] = (double)(N-i-1);
		}
		cudaMalloc(&gpu_x, N*sizeof(double)); 
		cudaMalloc(&gpu_y, N*sizeof(double));
		cudaMalloc(&gpu_z, N*sizeof(double));
		cudaMemcpy(gpu_x, x, N*sizeof(double), cudaMemcpyHostToDevice);
		cudaMemcpy(gpu_y, y, N*sizeof(double), cudaMemcpyHostToDevice);
		timer.reset();
		for (int n=0; n<5; n++) {
			addVec<<<param, param>>>(gpu_x, gpu_y, gpu_z, N);
			cudaDeviceSynchronize();
		}
		t_kernel += timer.get();
		printf("%d,%g\n", param, 0.2*t_kernel);
		cudaMemcpy(z, gpu_z, N*sizeof(double), cudaMemcpyDeviceToHost);
		cudaFree(gpu_x);
		cudaFree(gpu_y);
		cudaFree(gpu_z);
		free(x);
		free(y);
		free(z);
		k++;
	}
	return EXIT_SUCCESS;
}
\end{lstlisting}
\pagebreak

\section{CPP CUDA Code - Dot Product - Task 2a}
\label{app_2a}
\begin{lstlisting}[language=C++, title=C++ Listing]
#include <stdio.h>
#include "timer.hpp"

const int threads_per_block = 256;
double dot_cpu(double *a, double *b, int N) {
   double product = 0;
   for (int i = 0; i < N; i++)
   product = product + a[i] * b[i];
   return product;
}
__global__ void dotVec_one(double *x, double *y, double *partial_z, int N) {
	__shared__ double temp_arr[threads_per_block];
	double thread_product = 0;
	unsigned int global_tid = threadIdx.x + blockIdx.x * blockDim.x;
	unsigned int local_tid  = threadIdx.x;
	unsigned int total_threads = blockDim.x * gridDim.x;
	for (unsigned int i=global_tid; i<N; i+=total_threads) {
		thread_product += x[i] * y[i];
	}
	temp_arr[local_tid] = thread_product;
	for (unsigned int stride = blockDim.x/2; stride>0; stride/=2) {
		__syncthreads();
		if (threadIdx.x < stride) {
			temp_arr[threadIdx.x] += temp_arr[threadIdx.x + stride];
		}
	}
	if (threadIdx.x == 0) {
		partial_z[blockIdx.x] = temp_arr[0];
	}
}
__global__ void dotVec_two(double *partial_z) {
	for (int stride = blockDim.x/2; stride>0; stride/=2) {
		__syncthreads();
		if (threadIdx.x < stride)
			partial_z[threadIdx.x] += partial_z[threadIdx.x+stride];
	}
}
int main(void)
{
	// Task a //
	double *x, *y, *z;
	double *gpu_x, *gpu_y, *gpu_partial_z;
	Timer timer;
	int k = 0;
	int N_values_d[10] = { 100, 300, 1000, 3000, 10000, 30000, 100000, 300000, 1000000, 3000000 };
	printf("\nsize,time\n");
	while(k < 10) {
		int N = N_values_d[k];
		x = (double*)malloc(N*sizeof(double));
		y = (double*)malloc(N*sizeof(double));
		z = (double*)malloc(threads_per_block*sizeof(double));
		for (int i = 0; i < N; i++) {
			x[i] = 1.0;
			y[i] = 1.0;
		}
		cudaMalloc(&gpu_x, N*sizeof(double)); 
		cudaMalloc(&gpu_y, N*sizeof(double));
		cudaMalloc(&gpu_partial_z, threads_per_block*sizeof(double));
		cudaMemcpy(gpu_x, x, N*sizeof(double), cudaMemcpyHostToDevice);
		cudaMemcpy(gpu_y, y, N*sizeof(double), cudaMemcpyHostToDevice);
		timer.reset();
		for (int n=0; n<5; n++) { 
			dotVec_one<<<256, threads_per_block>>>(gpu_x, gpu_y, gpu_partial_z, N);	
			cudaDeviceSynchronize();
			dotVec_two<<<1, threads_per_block>>>(gpu_partial_z);
			cudaMemcpy(z, gpu_partial_z, threads_per_block*sizeof(double), cudaMemcpyDeviceToHost);
		}
		printf("%g,%g\n", z[0], 0.2*timer.get());
		cudaFree(gpu_x);
		cudaFree(gpu_y);
		cudaFree(gpu_partial_z);
		free(x);
		free(y);
		free(z);
		k++;	
	}
	return EXIT_SUCCESS;
}
\end{lstlisting}
\pagebreak

\section{CPP CUDA Code - Dot Product - Task 2b}
\label{app_2b}
\begin{lstlisting}[language=C++, title=C++ Listing]
#include <stdio.h>
#include "timer.hpp"

const int threads_per_block = 256;
double dotVec_cpu(double *a, double *b, int N) {
   double product = 0;
   for (int i = 0; i < N; i++)
   product = product + a[i] * b[i];
   return product;
}
__global__ void dotVec_gpu(double *x, double *y, double *partial_z, int N) {
	__shared__ double temp_arr[threads_per_block];
	double thread_product = 0;
	unsigned int global_tid = threadIdx.x + blockIdx.x * blockDim.x;
	unsigned int local_tid  = threadIdx.x;
	unsigned int total_threads = blockDim.x * gridDim.x;
	for (unsigned int i=global_tid; i<N; i+=total_threads) {
		thread_product += x[i] * y[i];
	}
	temp_arr[local_tid] = thread_product;
	for (unsigned int stride = blockDim.x/2; stride>0; stride/=2) {
		__syncthreads();
		if (threadIdx.x < stride) {
			temp_arr[threadIdx.x] += temp_arr[threadIdx.x + stride];
		}
	}
	if (threadIdx.x == 0) {
		partial_z[blockIdx.x] = temp_arr[0];
	}
}

int main(void)
{
	// Task b //	
	double *x, *y, z, *partial_z;
	double *gpu_x, *gpu_y, *gpu_partial_z;
	Timer timer;
	int k = 0;
	int N_values_d[10] = {100, 300, 1000, 3000, 10000, 30000, 100000, 300000, 1000000, 3000000};
	printf("\nsize,time\n");
	while(k < 10) {
		int N = N_values_d[k];
		x = (double*)malloc(N*sizeof(double));
		y = (double*)malloc(N*sizeof(double));
		partial_z = (double*)malloc(256*sizeof(double));
		for (int i = 0; i < N; i++) {
			x[i] = 1.0;
			y[i] = 1.0;
		}
		cudaMalloc(&gpu_x, N*sizeof(double)); 
		cudaMalloc(&gpu_y, N*sizeof(double));
		cudaMalloc(&gpu_partial_z, 256*sizeof(double));
		cudaMemcpy(gpu_x, x, N*sizeof(double), cudaMemcpyHostToDevice);
		cudaMemcpy(gpu_y, y, N*sizeof(double), cudaMemcpyHostToDevice);
		timer.reset();
		for (int n=0; n<5; n++) {
			dotVec_gpu<<<256, 256>>>(gpu_x, gpu_y, gpu_partial_z, N);
			cudaMemcpy(partial_z, gpu_partial_z, 256*sizeof(double), cudaMemcpyDeviceToHost);
			z = 0;
			for(int i=0; i<256; i++) {
				z += partial_z[i];
			}
		}
		printf("%d,%g\n", N, 0.2*timer.get());
		cudaFree(gpu_x);
		cudaFree(gpu_y);
		cudaFree(gpu_partial_z);
		free(x);
		free(y);
		free(partial_z);
		k++;
	}
	return EXIT_SUCCESS;
}
\end{lstlisting}
\pagebreak

\end{appendix}

