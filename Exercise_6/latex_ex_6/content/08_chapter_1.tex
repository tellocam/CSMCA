\section{Task1 : Inclusive and Exclusie Scan (1/4 Points)}
\subsection{Describe the Workings of the Kernels (1 Point)}
% wtf is actually going on:
% scan kernel 1: 256x256
% scan within each block and calculate/reduce there
% calculate workperthread and use according indices of the vector
% do the reduction:
% we do inclusive reduction -> then have to use...
% fancy ternary oparator to write 0 first -> threadIdx == 0 -> write 0, all thread write values, except last one
% write back result in vector
 
\subsubsection*{ workings of \fun{scan\_kernel\_1()}}
\begin{algorithm}
	\setstretch{1.5}
	\renewcommand{\thealgorithm}{}
	\caption{\fun{scan\_kernel\_1()} - Scan within each block} 
	\begin{algorithmic}[1]
	\State Call \fun{scan\_kernel\_1} \Comment{with \fun{<<<256,256>>>}}
	\State Kernel initiates \fun{shared\_buffer[256]} and \fun{my\_value}
    \State Calculate \fun{work\_per\_thread} from \fun{N} \Comment{Yields indices of vector chunk interfaces}
    \State Inc. scan for single vector chunk within \fun{shared\_buffer[256]} \Comment{concurrent for chunks}
    \State All threads write \fun{my\_value} except last one \Comment{\fun{(conditon) ? exp\_1 : exp\_2}}
    \State Write result to vector \fun{Y}
	\end{algorithmic}
\end{algorithm}

\subsubsection*{ workings of \fun{scan\_kernel\_2()}}
% scan kernel 2: (1x 256)
% add all the results of the kernels
% 1 block, 1 thread per block from (1)
% reduction over all the kernels
% write to output array accordingly again, s.t. we have exclusive scan
% -> 0 from threadIdx == 0, else write data at threadIdx-1

\begin{algorithm}
	\setstretch{1.5}
	\renewcommand{\thealgorithm}{}
	\caption{\fun{scan\_kernel\_2()} - Add results from Kernels} 
	\begin{algorithmic}[1]
	\State Call \fun{scan\_kernel\_2} \Comment{with \fun{<<<1,256>>>}}
	\State Reduction over all Kernels
	\State Write to output array \Comment{Exclusive Scan manner:}
	\State \fun{carries[threadIdx.x] = (threadIdx.x > 0) ? shared\_buffer[threadIdx.x - 1] : 0;}
	\end{algorithmic}
\end{algorithm}



\subsubsection*{workings of \fun{scan\_kernel\_3()}}
% scan kernel 3: (256x256)
% shared variable offset
% all the 0 threads of all blocks from (1):
% set offset as carries value from blockIdx
% add the offset to each value between blockstart and blockstop
%  we added all numbers together except the last one of the last thread

\begin{algorithm}
	\setstretch{1.5}
	\renewcommand{\thealgorithm}{}

	\caption{\fun{scan\_kernel\_3()}} 
	\begin{algorithmic}[1]
	\State call \fun{scan\_kernel\_3} \Comment{with \fun{<<<256,256>>>}}
	\State Kernel initiates \fun{shared\_offset}
    \State Calculate \fun{work\_per\_thread} from \fun{N} \Comment{Yields indices of vector chunk interfaces}
	\State $0^{\mathrm{th}}$ thread of every block set \fun{offset} as \fun{carries} from \fun{blockIdx.x}
	\State Add \fun{offset} to all values between \fun{blockstart} and \fun{blockstop}:
	\State \fun{Y[i] += shared\_offset;}

	\end{algorithmic}
\end{algorithm}

\pagebreak
\subsection{Provide an Implementation of Inclusive Scan (1 Point)}


\pagebreak
\subsection{Modify Exclusive Scan Code to Convert to Inclusive scan (1 Point)}


\pagebreak
\subsection{Compare Performances (1 Point)}


\pagebreak