\section{Task1 : Inclusive and Exclusie Scan (0/4 Points)}
\subsection{Describe the Workings of the Kernels (1 Point)}
wtf is actually going on:
scan kernel 1: 256x256
scan within each block and calculate/reduce there
calculate workperthread and use according indices of the vector
do the reduction:
we do inclusive reduction -> then have to use...
fancy ternary oparator to write 0 first -> threadIdx == 0 -> write 0, all thread write values, except last one
 
write back result in vector
 
scan kernel 2: (1x 256)
add all the results of the kernels
1 block, 1 thread per block from (1)
reduction over all the kernels
write to output array accordingly again, s.t. we have exclusive scan
-> 0 from threadIdx == 0, else write data at threadIdx-1
 
scan kernel 3: (256x256)
shared variable offset
all the 0 threads of all blocks from (1):
set offset as carries value from blockIdx
add the offset to each value between blockstart and blockstop
 we added all numbers together except the last one of the last thread




\begin{algorithm}
	\setstretch{1.5}
	\renewcommand{\thealgorithm}{}
	\caption{\texttt{scan\_kernel\_1}}
	\begin{algorithmic}[1]
    \State calculate \fun{work\_per\_thread} from \fun{N} \Comment{Yields indices of vector chunk interfaces}
    \State scan within each block \Comment{Done by reduction}
    \State Inclusive scan within each Buffer
    
    \State 
    
	\State \fun{resetDeformation}
	\State \fun{initializeParameters} $\alpha$, $\beta$, $\gamma$ \Comment{Aug. Lag. weights for
	$\mathrm{vol}(\Omega_{\mathrm{i}})$, $\mathrm{bc}_x(\Omega_{\mathrm{i}})$, $\mathrm{bc}_y(\Omega_{\mathrm{i}})$}
	\For{$\mathrm{i} < \mathrm{iter}_{\mathrm{max}}$}
		\State SolveStokes() \Comment{Solve Stokes on $\Omega_i$}
		\State SolveDeformationEquation() \Comment{Solve Auxiliary Problem on $\Omega_i$, yields $X$}
		\State Evaluate \fun{gfxbndnorm} = $ || X ||_{\mathrm{L}^2(\Gamma_{\infty , \mathrm{i}})} $
		\State Evaluate \fun{ScalingParamter} = $ \frac{0.01}{|| X ||_{\mathrm{L}^2(\Omega_{\mathrm{i}})}} $
		\If{\fun{gfxbndnorm} $<$ $\varepsilon$} 
			\State Increase $\alpha$, $\beta$, $\gamma$
		\If{\fun{parametersTooBig}}
			\State \fun{break}
		\EndIf
		\EndIf
		\State Set $\Omega_{i+1}$ =  $\Omega_i - X \cdot$ \fun{ScalingParameter} \Comment{Gradient Descent Step}
	\EndFor
	\end{algorithmic}
\end{algorithm}

\subsubsection*{scan\_kernel\_2}

\subsubsection*{scan\_kernel\_3}
\pagebreak
\subsection{Provide an Implementation of Inclusive Scan (1 Point)}


\pagebreak
\subsection{Modify Exclusive Scan Code to Convert to Inclusive scan (1 Point)}


\pagebreak
\subsection{Compare Performances (1 Point)}


\pagebreak